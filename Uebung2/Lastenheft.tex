\documentclass[11pt,a4paper]{scrartcl}
\usepackage[utf8]{inputenc}
\usepackage[pdftex]{hyperref}
\usepackage[T1]{fontenc}
\usepackage[ngerman]{babel}


\begin{document}

\section{Schnittstellen}
\paragraph{Web-Frontend}
Ein Web-Frontend soll die verschiedenen zur Verfügung stehenden Auswertungen und Statistiken (siehe \nameref{par:auswertung} und \nameref{par:statistiken}) auflisten und anzeigen.
\paragraph{API}
Sowohl die Auswertungen als auch eine Abgabe von Stimmen müssen über eine API zugänglich sein, so dass weitere Benutzeroberflächen entsprechende Funktionen bereitstellen können (z.B. elektronischer Stimmzettel).
\section{Funktionale Anforderungen}
\paragraph{Auswertung der Stimmen}
\label{par:auswertung}
Über die eingetragenen Stimmen soll die Sitzverteilung im Deutschen Bundestag über das Sainte-Laguë/Schepers Verfahren berechnet werden.
\paragraph{Statistiken}
\label{par:statistiken}
Es sollen Statistiken über die Stimmverteilung nach Bundesländern, Wahlkreisen, Parteien berechnet werden.
\paragraph{Weitere Analysen}
Liste mit Überhangmandaten, Kandidatenübersicht, Länderübersicht.
\paragraph{Hochrechnungen}
Es soll zudem möglich sein, bereits während der Wahlen erste Hochrechnungen für das Endergebnis zu ermitteln.
\paragraph{Dokumentation}
Dokumentation der API, sowie Inline-Kommentare.
\section{Nicht-Funktionale Anforderungen}
\paragraph{Korrektheit}
Alle Berechnungen sollen ein Korrektes Ergebnis liefern.
\paragraph{Performanz}
Da im Üblichen viele einzelne Stimmen/Stimmzettel verwaltet werden müssen, ist es wichtig, dass das System alle Berechnungen auf einer großen Datenmenge effizient durchführen kann.
\paragraph{Erweiterbarkeit}
Im Design ist darauf zu achten, dass später Stimmen abgegeben werden können, die dann mit in das System aufgenommen werden sollen.
\paragraph{Datenschutz}
Geltende Datenschutzrichtlinien müssen eingehalten werden (Wahlgeheimnis!).
\paragraph{Sicherheit}
Schutz vor Manipulation, etc.
\paragraph{Robustheit}
Abfangen von inkorrekten Benutzereingaben.
\section{Abnahmekriterien}
\begin{itemize}
\item Alle Funktionalen Anforderungen müssen implementiert und getestet sein.
\item Alle Nicht-Funktionalen Anforderungen müssen umgesetzt sein.
\end{itemize}
\end{document}